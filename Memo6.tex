\documentclass[a4paper]{article}
\usepackage[T1]{fontenc}
\usepackage{amsmath}
\usepackage{amssymb}
\usepackage{mathrsfs}
\usepackage[utf8]{inputenc}
\usepackage[cyr]{aeguill}
\usepackage[french]{babel}
\usepackage[pdftex]{graphicx}
\usepackage{amsfonts}
\usepackage{tabularx}
\usepackage[french,ruled,vlined]{algorithm2e} 
\usepackage{multirow}
\usepackage{pdflscape}
\usepackage{arydshln}
%\usepackage[counterclockwise]{rotating}


\usepackage{float} %% Pour placer une figure à un endroit précis sans qu'il puisse être déplacé

\newcounter{examplecounter}
\newenvironment{example}{
\begin{quote}%
    \refstepcounter{examplecounter}%
  \textbf{Example \arabic{examplecounter}}%
  \quad

\end{quote}%
}

\title{Algo}
\author{Digonaou KPEKPASSI}
\date{24 Mars 2015}

\begin{document}
Par rapport à l'algorithme de la semaine dernière nous avions échangé sur un problème concernant le cas où on a:\\
$x=a$ , $y=a$\\
$x=a$ , $y=d$\\
$x=e$ , $y=a$\\
$x=a$ , $y=e$\\
$x=f$ , $y=a$\\

On avait énumérer le fait que sur la première ligne

Voici un exemple simplifié afin de fournir des règles de préférence.
\begin{center}
\begin{tabular}{ |l|l|r| } 
\hline
 Label & Genre & Annee\\
 \hline
 \hline
T1	& Action	& 4\\
T2	& Aventure	& 4\\
T3	& Aventure	& 5\\
T4	& Action	& 3\\
T5	& Aventure	& 6\\
T6	& Action &	2\\

 \hline
\end{tabular}
\end{center}


 Le tableau des préférences fournit ici est:	
\begin{center}
\begin{tabular}{ |l|l| } 
T1 & T2\\
T1 & T3\\
T4 & T2\\
T1 & T5\\
T6 & T2\\
\end{tabular}
\end{center}
On suppose que l'année va de 2 à 7.

\begin{landscape}


\begin{tabular}{ |c|c|c|c|c|c|c||c|c|c|c|c|c|c|c|c|c| } 
\hline 
& & \multicolumn{5}{c|}{GENRE} &\multicolumn{10}{c|}{Annee}\\
\hline
 & & Action & Aventure & Comedie & $x=y$& $x\neq y$ & 2 & 3 & 4 & 5 & 6 & 7 & $x=y$ & $x\neq y$ & $x<y$ & $x>y$\\
\hline
  \multirow{2}{4em}{<T1,T2>}  & x & = & & &\multirow{2}{4em}{} & \multirow{2}{4em}{/} &  &  & = &  &  & & \multirow{2}{4em}{/} & \multirow{2}{4em}{} & \multirow{2}{4em}{} &\multirow{2}{4em}{} \\  
  \cdashline{3-5}							  &	y && = &&& &  & &=& & & &&&& \\
\hline


  \multirow{2}{4em}{<T1,T3>}  & x & = & & &\multirow{2}{4em}{} & \multirow{2}{4em}{/} &  &  & = &  &  & & \multirow{2}{4em}{} & \multirow{2}{4em}{/} & \multirow{2}{4em}{/} &\multirow{2}{4em}{} \\ \cdashline{3-5}
  							  &	y && = &&& &  & & &=& & &&&& \\
\hline

  \multirow{2}{4em}{<T4,T2>}  & x & = & & &\multirow{2}{4em}{} & \multirow{2}{4em}{/} &  & = &  &  &  & & \multirow{2}{4em}{} & \multirow{2}{4em}{/} & \multirow{2}{4em}{/} &\multirow{2}{4em}{} \\ \cdashline{3-5} 
  							  &	y && = &&& & & &=& & & &&&& \\
\hline

  \multirow{2}{4em}{<T1,T5>}  & x & = & & &\multirow{2}{4em}{} & \multirow{2}{4em}{/} & & & = & & & & \multirow{2}{4em}{} & \multirow{2}{4em}{/} & \multirow{2}{4em}{/} &\multirow{2}{4em}{} \\ \cdashline{3-5} 
  							  &	y && = &&& & & & & &=& &&&& \\                                     
\hline

  \multirow{2}{4em}{<T6,T2>}  & x & = & & &\multirow{2}{4em}{} & \multirow{2}{4em}{/} & = & & & & & & \multirow{2}{4em}{} & \multirow{2}{4em}{/} & \multirow{2}{4em}{/} &\multirow{2}{4em}{} \\ \cdashline{3-5} 
  							  &	y && = &&& & & &=& & & &&&& \\
\hline


\hline



\end{tabular}

Dans le cas de <T1,T2> nous remarquons que $x_{Annee}=y_{Annee}=4$. En supposant que le support est pris égal à 3, on peut obtenir la règle $C:\{x_{GENRE}\neq y_{GENRE}\}P:\{x_{GENRE}=Action,x_{Annee}=4,y_{GENRE}=Aventure\}$. Cette règle ne serait pas prise en compte si on considérait que les items $x=4$ et $y=4$ ne peuvent pas être pris séparément dans <T1,T2> étant donné que ces vleurs sont égales. 

\end{landscape}


\end{document}